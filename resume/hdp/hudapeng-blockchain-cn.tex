%%%%%%%%%%%%%%%%%%%%%%%%%%%%%%%%%%%%%%%%%%%%%%
% Document Header
% Chinese version should be built with XeLaTex
%%%%%%%%%%%%%%%%%%%%%%%%%%%%%%%%%%%%%%%%%%%%%%

\documentclass[10pt,a4paper]{moderncv}


% optional argument are 'blue' (default), 'orange', 'red', 'green', 'grey' and 'roman' (for roman fonts, instead of sans serif fonts)
\moderncvtheme[blue]{classic}                % idem  casual,  oldstyle,  banking, classic

% Configuration for Chinese
\usepackage[utf8]{inputenc}
\usepackage[slantfont,boldfont]{xeCJK}
\setCJKmainfont{STSong}
%\setCJKmainfont{SimSun}
%\setCJKfamilyfont{song}{SimSun}

%%\usepackage{graphicx}
% adjust the page margins
\usepackage[scale=0.8]{geometry}
\usepackage{verbatim}
% %\usepackage{graphicx, listings, multicol, wrapfig, caption, subfig, color, }

%\quote{``Do what you fear, and the death of fear is certain.''\\-- Anthony Robbins}
\AtBeginDocument{\recomputelengths}  % required when changes are made to page layout lengths
% personal data
\firstname{胡大鹏}
\familyname{}
%\title{区块链极客, 投资者}
\mobile{+86 18612523725}
\social[linkedin][cn.linkedin.com/in/hudapeng]{LinkedIn/hudapeng}
\social[github][github.com/dapenghu]{Github/hudapeng}
\email{damon.hu@hotmail.com}
\photo[70pt]{images/me.png}
% '64pt' is the height the picture must be resized

%\usepackage{hyperref}
%\hypersetup{    colorlinks=true,    linkcolor=blue,    filecolor=magenta,  urlcolor=cyan}
\AfterPreamble{
	\hypersetup{ colorlinks=true,  linkcolor=cyan,  filecolor=magenta,    urlcolor=blue}
}
\nopagenumbers{}       

% uncomment to suppress automatic page numbering for CVs longer than one page
%% \renewcommand*{\sectionfont}{\LARGE\sffamily\monospace\slshape}
%% \renewcommand*\addressfont{\fontfamily{pzc}\selectfont}
%% \renewcommand*\sectionfont{\fontfamily{pzc}\fontsize{20}{24}\selectfont}

%%%%%%%%%%%%%%%%%%%%%%%%%%%%%%%%%%%%%
% Document Body
%%%%%%%%%%%%%%%%%%%%%%%%%%%%%%%%%%%%%

\begin{document}
\maketitle

%\section{Master thesis}
%\cvline{title}{\emph{Title}}
%\cvline{supervisors}{Supervisors}
%\cvline{description}{\small Short thesis abstract}

%%========================== Highlight ================================
\section{简介}
\vspace*{0.2\baselineskip}
\cventry{学历背景}{毕业于北京大学,数学硕士学位}{}{}{}{}
\cventry{编程语言}{Java, C/C++,  C\raisebox{.6ex}{\scriptsize\#}, GoLang, Javascript, Python}{}{}{}{}
\cventry{区块链技术}{比特币, 以太坊, 密码学, 智能合约, 共识算法}{}{}{}{}
\cventry{工作经历}{16年IT$\backslash$互联网公司工作经验: 微软, 百度, 甲骨文, 华为}{}{}{}{}
\cventry{行业经验}{通信,中间件,云计算,大数据,互联网,音视频}{}{}{}{}
%\cventry{投资经历}{10个月加密货币和ICO 投资经验,曾经参投: OMG, Tenx, BTM, Status等}{}{}{}{}
%\cventry{创业经历}{两家区块链创业公司: AskCoin, 万维链}{}{}{}{}

%%========================== Education ================================
\vspace*{0.4\baselineskip}

\section{学历}
\vspace*{0.2\baselineskip}
\cventry{
	\begin{minipage}[p]{8mm}
		\includegraphics[width=8mm]{images/pku}
	\end{minipage}
\hspace{15 pt}}{北京大学}{应用数学硕士}{1998-09 $\sim$ 2001-06}{}{}


\vspace*{0.2\baselineskip}
\cventry{
	\begin{minipage}[p]{8mm}
		\includegraphics[width=8mm]{images/sdu}
	\end{minipage}
\hspace{14 pt}}{山东大学} {计算数学学士}{1994-09 $\sim$ 1998-07}{}{}

%%========================== Project Experience ================================
\vspace*{0.4\baselineskip}

\section{工作经验}
\vspace{2ex}
%%========================== OKGroup ================================
\cventry{
    \begin{minipage}[p]{14mm}
        \includegraphics[width=10mm]{images/ok}
\end{minipage}}	{OK区块链资本 -- 首席研究员}{}{}{}{}

\vspace{1ex}
\cventry{2018-02  \linebreak $\sim$ 至今}{OK区块链资本}{}{}{}
{
    OK集团旗下区块链领域的风险投资基金。OK集团经过6年的发展,已经建⽴了 OKCoin 数字资产交易易平台、OKCoinKr 数字资产交易易平台、OKLink 全球跨境⽀付⽹网络、OK区块链工程院,并拥有OKEx数字资产及衍⽣品交易平台等。%积累了丰富的区块链研究及技术开发经验,在区块链产业投资布局及战略合作方面展现出突出的前瞻性。
    \begin{itemize}
        \item[-] 筹备稳定币项目 USDK,撰写白皮书、项目可行性分析、合规框架设计。
        \item[-] 研究区块链在金融领域的应用,梳理美联储、欧洲央行、DTCC等权威机构的研究报告,撰写《Libra监管与合规白皮书》。
        \item[-] 总结并且跟踪区块链技术发展,包括虚拟机、共识算法、治理机制、二层清算协议等。
        \item[-] 主导与中科院、五道口金融研究院等学术机构合作。
        \item[-] 为投资与交易所上币业务提供技术咨询和项目评估
    \end{itemize}
}
\vspace{2ex}

%%========================== Microsoft ================================
\cventry{
	\begin{minipage}[p]{14mm}
		\includegraphics[width=10mm]{images/ms}
	\end{minipage}}	{微软 -- 架构师}{Level 63}{}{}{}

\vspace{1ex}
\cventry{2016-07  \linebreak $\sim$ 2017-06}{\href{https://azure.microsoft.com/} {Azure} - Data Lake Store \& Analytics}{云计算、大数据}{}{}
{
    \textbf{ADLA} - Azure Data Lake Analytics 是一个经济、高效、易操作的流式大数据分析云服务。用户只需要专注于编写、运行和管理作业,而不需要管理底层的大数据基础设施。ADLA的特性包括: 动态扩容、可视化执行、按需付费、自动部署、丰富的查询语言、多种数据存储系统。
    \begin{itemize}
		\item[-] 优化SQL server 数据库设计,提高SQL运行效率,提高系统的可靠性
		\item[-] 优化认证与授权系统,支持细粒度SQL访问策略
	\end{itemize}
}
\vspace{2ex}

\pagebreak

%%=========================  Baidu  ================================
\cventry{
	\begin{minipage}[p]{14mm}
		\includegraphics[width=10mm]{images/baidu.jpg}
	\end{minipage} }
	{ 百度 -- 架构师}{T7}{}{}{}
	\vspace{1ex}

\cventry{2015-03 \linebreak $\sim$ 2016-06}{\href{https://cloud.baidu.com} {百度云}}{云计算,流媒体}{}{}
{
	 依托于百度高可靠、高性能、低成本、大容量的数据存储、内容分发、GPU计算等基础设施,多媒体智能云为企业提供了视频点播、视频直播、音视频转码、图像分析、内容审核等多媒体服务.
	 \begin{itemize}
		\item[-] 设计多媒体转码服务,支持大规模并行处理转码任务.
		\item[-] 优化视频直播服务的性能,通过自适应码率、缓存优化,容错等技术把8-10秒的延时降低至3-5秒。
	 \end{itemize}
}

\vspace{2ex}
\vspace*{0.4\baselineskip}

%%=========================  Oracle  ================================
\cventry{
    \begin{minipage}[p]{14mm}
        \includegraphics[width=15mm]{images/oracle}
   \end{minipage} }
{ 甲骨文 -- 资深软件工程师}{企业应用平台、中间件}{}{}{}
\vspace{1ex}
\cventry{2011-02 \linebreak $\sim$ 2014-11}{\href{https://www.oracle.com/middleware/weblogic/index.html}{WebLogic/GlassFish 应用服务器}}{JavaEE, 云计算}{}{}
{
  WebLogic与GlassFish分别是商用和开源的 Java 企业应用服务器。它们都支持标准 JavaEE 规范,提供安全、可靠、高性能、可扩容的企业应用部署平台。
  \begin{itemize}
    \item[-] 作为JCA (Java Connector Architecture) 专家组成员,参与 JCA 1.7 规范的制定和参考实现。
    \item[-] 领导 JCA 容器的设计和开发,实现虚拟容器特性,支持应用服务器的PaaS功能.
  \end{itemize}
}

%%==========================  Nortel  ===============================
\vspace{2ex}
\vspace*{0.4\baselineskip}

\cventry{
    \begin{minipage}[p]{8mm}
            \includegraphics[width=8mm]{images/nortel}
    \end{minipage}}
{北电网络 -- 资深软件工程师}{通信}{}{}{}
\cventry{2008-01 \linebreak $\sim$ 2010-10}{Agile Communication Environment (ACE)}{Java, Web服务}{}{}
{
  基于 ParlayX 规范,ACE将通信系统的数据与功能包装成Web服务,为企业应用提供了通信系统的系统集成平台,
  \begin{itemize}
    \item[-] 设计并实现Web应用的性能管理功能,收集、监控各个服务的性能指标,实时发现故障并告警 \end{itemize}
}

\vspace*{0.2\baselineskip}
\cventry{2003-09 \linebreak $\sim$ 2007-12}{Computer Aided Testing Tool (CATT)}{SS7, 3GPP}{}{}
{
  CATT 是一个多功能通信协议测试平台。它支持SS7, 3GPP, ITU-T等多种协议族,提供易用灵活的测试脚本开发环境,和大规模并行测试运行环境。
  \begin{itemize}
    \item[-] 面向3G 核心网络设备,设计并实现多种协议的测试组件,例如H.248, Q.2630 等。
  \end{itemize}
}

%%============================  Huawei  =============================
\vspace*{0.2\baselineskip}
\vspace*{0.2\baselineskip}
\vspace*{0.4\baselineskip}

\cventry{
    \begin{minipage}[p]{8mm}
            \includegraphics[width=8mm]{images/huawei}
    \end{minipage}}
{华为北研所 -- 软件工程师}{通信}{}{}{}

\cventry{2001/07 --2003/08}{UMG8900 多媒体网关}{C, VxWorks}{}{}
{
  UMG8900 多媒体网关 是GSM/UMTS 核心网络设备. 它负责IP、ATM,TDM 多种传输网络之间的数据交换和互通。
  \begin{itemize}
    \item[-] 设计并实现IP、ATM、TDM、TC的资源管理和负载均衡功能。
  \end{itemize}
}

%%=========================================================
\vspace*{0.4\baselineskip}


\section{技术资格认证}
\cventry{
    \begin{minipage}[p]{9mm}
        \includegraphics[width=9mm]{images/scea}
    \end{minipage}
}
{Sun认证Java企业架构师 }{}{}{}{}
\vspace*{0.4\baselineskip}
\cventry{
    \begin{minipage}[p]{9mm}
        \includegraphics[width=9mm]{images/scjp}
    \end{minipage}
}
{Sun认证Java程序员 }{}{}{}{}

% \closesection{}                   % needed to renewcommands
% \renewcommand{\listitemsymbol}{-} % change the symbol for lists

%%========================== Startup ================================
\vspace*{0.4\baselineskip}
\section{区块链创业经验}
\vspace*{0.2\baselineskip}
\cventry{
    \begin{minipage}[p]{14mm}
        \includegraphics[width=10mm]{images/ask}
\end{minipage} }
{ AskCoin -- 核心开发者}{2017-07 $\sim$ 2017-09}{}{}
{AskCoin 是一个面向知识共享服务的区块链平台. AskCoin 基于DAG(有向无环图)技术重新实现共识账本系统. 相对于比特币, 以太坊等以区块链为基本数据结构的共识账本系统,DAG在响应延时,吞吐量等系统性能上有很大的优势,非常适合大量的小额支付场景}
\vspace{1ex}

\cventry{
    \begin{minipage}[p]{14mm}
        \includegraphics[width=10mm]{images/wanchain}
\end{minipage} }
{万维链 -- 架构师}{2017-11 $\sim$ 2017-12}{}{}
{万维链旨在建立一个通用的跨链协议,任何区块链网络,无论公有链、还是联盟链,均能低成本的接入万维链。不但能完成数字资产的跨链交易, 还能支持跨链的智能合约。同时应用最新的密码学技术,保证资产交易和智能合约的隐私性}
\vspace{1ex}

\end{document}
